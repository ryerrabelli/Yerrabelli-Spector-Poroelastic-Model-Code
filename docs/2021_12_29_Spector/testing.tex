It seems that finding convenient-to-analyze values of $C_0$, $C_1$, $C_2$ will not be relevant.

$ \text{For } C_0 = \frac{C_{11}-C_{12}}{C_{11}}=1 \text{, then } C_{12}=0  \\
\text{Set } C_{12} = \frac{E_1 \left(\nu_{21}+\nu_{31}^2\frac{E_1}{E_3}\right)}{\left(1+\nu_{21} = 0\right)\Delta_1}   \\
\text{Thus, } E_1=0 \text{  or  } \nu_{21}+\nu_{31}^2\frac{E_1}{E_3}=0 \\
\text{Since } E_1 => C_{11}=C_{12}=0, \text{, which make} C_0, C_1, C_2 \text{ undefined, so this is not possible.} \\
\text{If we assume all parameters are non-negative i.e. } x \geq 0 \text{ for } x \in \left\{\nu_{21}, \nu_{31}, E_1, E_3 \right\} , \\  \text{then this is only possible if } \nu_{21}=0=\nu_{31}.
$
If we ass



\subsection{How to change the margins and paper size}

Usually the template you're using will have the page margins and paper size set correctly for that use-case. For example, if you're using a journal article template provided by the journal publisher, that template will be formatted according to their requirements. In these cases, it's best not to alter the margins directly.

If however you're using a more general template, such as this one, and would like to alter the margins, a common way to do so is via the geometry package. You can find the geometry package loaded in the preamble at the top of this example file, and if you'd like to learn more about how to adjust the settings, please visit this help article on \href{https://www.overleaf.com/learn/latex/page_size_and_margins}{page size and margins}.